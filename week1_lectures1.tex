%%%%%%%%%%%%%%%%%%%%%%%%%%%%%%%%%%%%%%%%%%%%%%%%%%%%%%%%%%%%%%%%%%%%%%%%%%%%%

\begin{edXsection}{Lectures U1.1: History and development of quantum computation}

%%%%%%%%%%%%%%%%%%%%

\begin{edXvertical}{About Unit 1}

\begin{edXtext}{About Unit 1}

{\LARGE About Unit 1}

{\noindent\bf \underline{Lecture Topics:}} Introduction to quantum
computation; classical Boolean logic; introduction to quantum
mechanics; quantum wierdness.

We begin this course with an overview of quantum computation, then
build a foundation for the semester by describing the basic tenets of
classical computing, based on Boolean logic.  We then present the
fundamental principles of quantum mechanics, in the context of quantum
computation, based on four postulates governing states, time-evolution
of states, measurement, and tensor products of states.  We then
illustrate some of the strangest, non-classical features of quantum
mechanics, many of which arise in the context of a property known as
quantum entanglement.

~\\
{\noindent\bf \underline{Optional Reading:}}
\href{https://www.amazon.com/Quantum-Computation-Information-10th-Anniversary/dp/1107002176}{Nielsen and Chuang}, Chapter 1

% \edXaskta{settings=1 to=csamolis@mit.edu cc=ichuang@mit.edu}

\end{edXtext}

\AddSearchBox{1}

\end{edXvertical}

%%%%%%%%%%%%%%%%%%%%

\begin{edXvertical}{Course topics for the semester}

\edXvideo{Course topics for the semester}{mkEzEkrqrlk}[url_name=U1L1a]

\AddSearchBox{2}

\end{edXvertical}

%%%%%%%%%%%%%%%%%%%%

\begin{edXvertical}{Models for classical computation}

\edXvideo{Models for classical computation}{BM4hlDQsc1M}[url_name=U1L1b]

% \edXvideo{Density matrices}{37BFmLp8TTc}[url_name=Q1L2a]

\AddSearchBox{3}

\end{edXvertical}

%%%%

\begin{edXvertical}{CQ: Models for classical computation}

\begin{edXproblem}{Universality of classical circuits}{url_name=u1-1-cq-universality attempts=1}

This is a concept question, provided to double-check your understanding of the previous video clip.

In lecture, Prof. Shor introduced the topic of universality, and
described how some models of classical computation which are
universal, and other models are not.

The circuit model of classical computation may be defined as the
family of electrical circuits composed from AND gates and NOT gates.

Check all the following which are true for the circuit model:

\edXabox{type="oldmultichoice"
    expect="Circuits are universal for classical computation","Circuits can describe computations which are beyond what a Turing machine can do","Circuits can simulate Turing machines"
    options="Circuits are universal for classical computation","Circuits are NOT a universal model for classical computation","Circuits can describe computations which are beyond what a Turing machine can do","Circuits can simulate Turing machines"
}

\end{edXproblem}

\AddSearchBox{1}

\end{edXvertical}

%%%%%%%%%%%%%%%%%%%%

\begin{edXvertical}{History of quantum mechanics and quantum computation}

\edXvideo{History of quantum mechanics and quantum computation}{Fm{-}gIjGlpBo}[url_name=U1L1c]

\AddSearchBox{4}

\end{edXvertical}

%%%%

\begin{edXvertical}{CQ: Faster-than-light communication with entanglement?}

\begin{edXproblem}{Faster-than-light communication with entanglement?}{url_name=u1-1-cq-qm-flash attempts=1}

This is a concept question, provided to double-check your understanding of the previous video clip.

As described by Prof. Shor in the last lecture video clip, in 1982,
Nick Herbert published
\href{https://link.springer.com/article/10.1007\%2FBF00729622}{a paper
  proposing FLASH}, ``First Laser-Amplified Superluminal Hookup,''
offering a means to communicate faster than the speed of light, using
quantum entanglement.

What was the flaw in Herbert's proposal?

\edXabox{type="multichoice"
    expect="Unknown quantum states cannot be copied"
    options="Entangled states have too short of a lifetime to allow superluminal communication","Entangled photons are states of light which cannot travel faster than the speed of light","Quantum entanglement involves hidden variables, which obfuscate any superluminal communication","Unknown quantum states cannot be copied"
}

\end{edXproblem}

\AddSearchBox{5}

\end{edXvertical}

%%%%%%%%%%%%%%%%%%%%

\begin{edXvertical}{Computation must be robust against noise}

\edXvideo{Computation must be robust against noise}{OFWyhxzLN7k}[url_name=U1L1d]

\AddSearchBox{6}

\end{edXvertical}

%%%%

\begin{edXvertical}{CQ: Mechanisms for fault-tolerance}

\begin{edXproblem}{Mechanisms for fault-tolerance}{url_name=u1-1-cq-qm-flash attempts=1}

This is a concept question, provided to double-check your understanding of the previous video clip.

Which of the following are mechanisms for fault-tolerance, as used in classical computing?  Check all which apply:

\edXabox{type="oldmultichoice"
    expect="Massive redundancy","Checkpointing","Error correction"
    options="Massive redundancy","Checkpointing","Code rewriting","Error correction"
}

\end{edXproblem}

\AddSearchBox{7}

\end{edXvertical}

%%%%%%%%%%%%%%%%%%%%

\end{edXsection}
