%%%%%%%%%%%%%%%%%%%%%%%%%%%%%%%%%%%%%%%%%%%%%%%%%%%%%%%%%%%%%%%%%%%%%%%%%%%%%
%
% File:    8371-spring-2008-qc-class-syllabus.tex
% Date:    01-Feb-08
% Authors: I. Chuang <ichuang@mit.edu>
%
%%%%%%%%%%%%%%%%%%%%%%%%%%%%%%%%%%%%%%%%%%%%%%%%%%%%%%%%%%%%%%%%%%%%%%%%%%%%%

\documentclass[preprint,pra,12pt]{ikedoc3}
\usepackage{latexsym}
\usepackage{amsfonts}

%%%%%%%%%%%%%%%%%%%%%%%%%%%%%%%%%%%%%%%%%%%%%%%%%%%%%%%%%%%%%%%%%%%%%%%%%%%%%
% Macros

%%%%%%%%%%%%%%%%%%%%%%%%%%%%%%%%%%%%%%%%%%%%%%%%%%%%%%%%%%%%%%%%%%%%%%%%%%%%%
\begin{document}

%\thispagestyle{empty}
%\pagestyle{empty}

%\centerline{\Large DRAFT}
\centerline{\large MASSACHUSETTS INSTITUE OF TECHNOLOGY}
\vspace*{1ex}
\centerline{\large Department of Physics \& Department of Applied Math}
\vspace*{2ex}

\centerline{\large MIT 6.443J \,/\, 8.371J \,/\, 18.409 \,/\, MAS.865}
\vspace*{1ex}

\centerline{\large Quantum Information Science}
\vspace*{1ex}
\centerline{\large February 5, 2008}

~\\
\centerline{\large \bf General Information \& Syllabus}

\vspace*{4ex}

%%%%%%%%%%%%%%%%%%%%%%%%%%%%%%%%%%%%%%%%%%%%%%%%%%%%%%%%%%%%%%%%%%%%%%%%%%%%%
{\noindent\bf \underline{General Information:}} 

Prereq.: 2.111 / 18.435J / ESD.79
\par
Units: 3-0-9

\begin{quote}
Advanced graduate course on quantum computation and quantum
information.  Prior knowledge of quantum mechanics and basic
information theory is required.  The first semester of this two-course
sequence (2.111 / 18.435J ) was taught by Seth Lloyd in the Fall of
2007, and covered quantum algorithms, quantum error correction,
cryptography, and introduced fault tolerance.  This semester, we will
cover models of quantum computation, advanced quantum error correction
codes, fault tolerance, quantum algorithms beyond factoring,
properties of quantum entanglement, and quantum protocols.

\noindent
% {\bf NB:} This class will be closely coordinated with a complimentary
% course, Physics 287, given by Prof. Mikhail Lukin at Harvard, that
% will emphasize physical implementations and can be an excellent
% addition.  The Harvard class meets in Jefferson 453, MW 11:30-1pm.
\end{quote}

\begin{tabbing}
Instructors:~ \= Geva Patz, E15-435 {\tt <geva@media.mit.edu>}; \kill
Lectures: \> Tuesday \& Thursday 11am-12:30pm, Room 36-153 \\
Instructors: \> Prof. Isaac Chuang, 26-251 {\tt <ichuang@mit.edu>} \\
             \> Prof. Peter Shor, 2-284 {\tt <shor@math.mit.edu>} ;
	office hours by appointment
\\
TA: \>  Andrew Cross {\tt <awcross@mit.edu>}\\
% Secretary: \> Murray Whitehead, E15-435, {\tt murrayw@media.mit.edu}\\
Textbook: \> Quantum Computation and Quantum Information, by Nielsen and Chuang
\\
Grading: \> Homework (4 problem sets) 40\%, Project presentation 20\%
	    Project paper 40\% 
\\
Schedule: \> Final project paper due on May 15, 2008
\\
Web site: \> {\tt http://web.mit.edu/cua/www/8.371}
\end{tabbing}

%%%%%%%%%%%%%%%%%%%%%%%%%%%%%%%%%%%%%%%%%%%%%%%%%%%%%%%%%%%%%%%%%%%%%%%%%%%%%
\vspace*{0.5cm}

{\noindent\bf \underline{Syllabus:}} 

\begin{itemize}
\item[\bf [T 05-Feb]] Lecture 1: General introduction; Quantum operations
\item[\bf [R 07-Feb]] Lecture 2: Quantum error correction - criteria
and examples 
                        \hfill \mbox{[PS\#1 out] }

\item[\bf [T 12-Feb]] Lecture 3: Calderbank Shor Steane codes 

\item[\bf [R 14-Feb]] Lecture 4: Stabilizers ; stabilizer quantum
codes  \hfill \mbox{[PS\#2 out, PS\#1 due]}

\item[\bf [T 19-Feb]] No class (Monday schedule) 

\item[\bf [R 21-Feb]] Lecture 5: CWS codes and nonabelian codes

\item[\bf [T 26-Feb]] Lecture 6: Stabilizers II ; computing on quantum
codes 

\item[\bf [R 28-Feb]] Lecture 7: concatenated codes ; the threshold
theorem \hfill \mbox{[PS\#3 out, PS\#2 due] }

\item[\bf [T 04-Mar]] Lecture 8: Cluster state quantum computation 

\item[\bf [R 06-Mar]] Lecture 9: Measurement and teleportation based
quantum computation 

\item[\bf [T 11-Mar]] Lecture 10: Adiabatic quantum computatin 

\item[\bf [R 13-Mar]] Lecture 11: Qauntum algorithms on graphs;
quantum random walks  \hfill \mbox{[PS\#4 out, PS\#3 due]}

\item[\bf [T 18-Mar]] Lecture 12: Quantum algorithms: the abelian
hidden subgroup problem ; QFT over Sn 

\item[\bf [R 20-Mar]] Lecture 13: The nonabelian HSP ; hidden dihedral
group ; positive and negative results 

\item[\bf [T 25-Mar]] Spring Break
 

\item[\bf [R 29-Mar]] Spring Break
 

\item[\bf [T 01-Apr]] Lecture 14: Channels I: Quantum data
compression; entanglement concentration; typical subspaces 

\item[\bf [R 03-Apr]] Lecture 15: Channels II: Holevo's theorem ; HSW
theorem ; entanglement assisted channel capacity 
\hfill \mbox{[Project forms out, PS\#4 due]}

\item[\bf [T 08-Apr]] Lecture 16: Channels III: quantum-quantum
channels, mother/father protocol ; distillable entanglement 

\item[\bf [R 10-Apr]] Lecture 17: Entanglement as a physical resource 

\item[\bf [T 15-Apr]] Lecture 18: Quantum games 

\item[\bf [R 17-Apr]] Lecture 19: Quantum protocols - quantum
communicatin complexity ; distributed algorithms  
\hfill\mbox{[Project forms due]}

\item[\bf [T 22-Apr]] MIT Holiday: Patriot's day 

\item[\bf [R 24-Apr]] Lecture 20: Quantum cryptography 

\item[\bf [T 29-Apr]] Project meetings 

\item[\bf [R 01-May]] Project meetings 

\item[\bf [T 06-May]] Project presentations 

\item[\bf [R 08-May]] Project presentations 

\item[\bf [T 13-May]] Project presentations 

\item[\bf [R 15-May]] All final project papers due

\end{itemize}

\end{document}
