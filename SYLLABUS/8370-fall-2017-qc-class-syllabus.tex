%%%%%%%%%%%%%%%%%%%%%%%%%%%%%%%%%%%%%%%%%%%%%%%%%%%%%%%%%%%%%%%%%%%%%%%%%%%%%
%
% File:    8370-spring-2017-qc-class-syllabus.tex
% Date:    06-Sep-2017
% Authors: I. Chuang <ichuang@mit.edu> / P. Shor <shor@math.mit.edu>
%
%%%%%%%%%%%%%%%%%%%%%%%%%%%%%%%%%%%%%%%%%%%%%%%%%%%%%%%%%%%%%%%%%%%%%%%%%%%%%

\documentclass[preprint,pra,12pt]{ikedoc3}
\usepackage{latexsym}
\usepackage{amsfonts}

%%%%%%%%%%%%%%%%%%%%%%%%%%%%%%%%%%%%%%%%%%%%%%%%%%%%%%%%%%%%%%%%%%%%%%%%%%%%%
% Macros

%%%%%%%%%%%%%%%%%%%%%%%%%%%%%%%%%%%%%%%%%%%%%%%%%%%%%%%%%%%%%%%%%%%%%%%%%%%%%
\begin{document}

%\thispagestyle{empty}
%\pagestyle{empty}

%\centerline{\Large DRAFT}
\centerline{\large MASSACHUSETTS INSTITUE OF TECHNOLOGY}
\vspace*{1ex}
\centerline{\large Department of Physics \& Department of Applied Math}
\vspace*{2ex}

\centerline{\large MIT 18.435 \,/\, 8.370 \,/\, 2.111}
\vspace*{1ex}

\centerline{\large Quantum Information Science I}
\vspace*{1ex}
\centerline{\large September 7, 2017}

~\\
\centerline{\large \bf General Information \& Syllabus}

\vspace*{4ex}

%%%%%%%%%%%%%%%%%%%%%%%%%%%%%%%%%%%%%%%%%%%%%%%%%%%%%%%%%%%%%%%%%%%%%%%%%%%%%
{\noindent\bf \underline{General Information:}} 

% Prereq.: 2.111 / 18.435J / 8.370
% \par
Units: 3-0-9

\begin{quote}
Provides an introduction to the theory and practice of quantum
computation. Topics covered: physics of information processing;
quantum logic; quantum algorithms including Shor's factoring algorithm
and Grover's search algorithm; quantum error correction; quantum
communication and cryptography. Prior knowledge of quantum mechanics
helpful but not required. First course in a sequence of two core
quantum information science courses at MIT.

\noindent
% {\bf NB:} This class will be closely coordinated with a complimentary
% course, Physics 287, given by Prof. Mikhail Lukin at Harvard, that
% will emphasize physical implementations and can be an excellent
% addition.  The Harvard class meets in Jefferson 453, MW 11:30-1pm.
\end{quote}

\begin{tabbing}
Instructors:~ \= Geva Patz, E15-435 {\tt <geva@media.mit.edu>}; \kill
Lectures: \> Tuesday \& Thursday 1pm-2:30pm, Room 4-370 \\
Instructors: \> Prof. Peter Shor, 2-375 {\tt <shor@math.mit.edu>}  \\
             \> Prof. Isaac Chuang, 26-251 {\tt <ichuang@mit.edu>};   \\
	\> office hours -- see Stellar page
\\
TA: \>  Ryuji Takagi {\tt <rtakagi@mit.edu>}\\
% Secretary: \> Murray Whitehead, E15-435, {\tt murrayw@media.mit.edu}\\
Textbook: \> Quantum Computation and Quantum Information, by Nielsen and Chuang
\\
Grading: \> Homework (8 problem sets) 40\%, Midterm exam 20\%
	    Final exam 30\%, Participation 10\%
\\
Schedule: \> Midterm exam -- Tuesday October 24, 2017
\\
Web site: \> {\tt https://lms.mitx.mit.edu/courses/course-v1:MITx+8.370r+2017\_Fall/about}
\end{tabbing}

%%
\paragraph{Late Policy.} Problem sets will due online at 7pm.
It is possible to hand in a problem set late, with a 10\% decrease 
in your grade, until 7pm on the following day. 
We will not accept assignments turned in after the 
solutions are posted without a note from S$^3$.

\paragraph{Collaboration Policy.} Collaboration on homework is permitted, 
you should first think about the problems on your own. 
You must list the names of your collaborators on your submitted homework, 
or state that you had none. 

\paragraph{Student Support Services.} If you are dealing with a 
personal or medical issue that is impacting your ability to attend 
class, complete work, or take one of the exams, please discuss this 
with Student Support Services (S$^3$). The deans in S$^3$ will 
verify your situation, and then discuss with you how to address the
missed work. Students will not be excused from coursework or be 
given credit for homework more than one day late without verification 
from Student Support Services. You may consult with S$^3$  in 5-104 or 
at 617-253-4861. Also, S$^3$ has walk-in hours Monday-Friday 9:00--10:00am.


%%%%%%%%%%%%%%%%%%%%%%%%%%%%%%%%%%%%%%%%%%%%%%%%%%%%%%%%%%%%%%%%%%%%%%%%%%%%%
\vspace*{0.5cm}

{\noindent\bf \underline{Syllabus:}} 

\vspace*{1ex}

\noindent
{\bf Unit 1: Quantum and classical computing fundamentals}

\begin{itemize}
  \setlength\itemsep{0.1em}
\item[\bf [9/7]] Lecture 1.1: Introduction to quantum computation
\item[\bf [9/12]] Lecture 1.2: Classical logic
\item[\bf [9/14]] Lecture 1.3: Intro to QM - states
\item[\bf [9/19]] Lecture 1.4: Intro to QM - time evolution
\item[\bf [9/21]] Lecture 1.5: Intro to QM - measurement and tensor products
\item[\bf [9/26]] Lecture 1.6: Quantum weirdness
\end{itemize}

\noindent
{\bf Unit 2: Simple quantum protocols and algorithms}

\begin{itemize}
  \setlength\itemsep{0.1em}
\item[\bf [9/28]] Lecture 2.1: Teleportation and superdense coding
\item[\bf [10/3]] Lecture 2.2: Quantum Circuit Model and Deutsch-Jozsa
\item[\bf [10/5]] Lecture 2.3: Simons algorithm
\item[\bf [10/10]] Lecture 2.4: no class (Columbus day)
\item[\bf [10/12]] Lecture 2.5: Quantum Fourier Transform
\item[\bf [10/17]] Lecture 2.6: Phase estimation
\item[\bf [10/19]] Lecture 2.7: Shor's algorithm
\end{itemize}

\begin{itemize}
\item[\bf [10/24]] Midterm exam
\end{itemize}

\noindent
{\bf Unit 3: Quantum noise, codes and communication}

\begin{itemize}
  \setlength\itemsep{0.1em}
\item[\bf [10/26]] Lecture 3.1: Classical error correcting codes
\item[\bf [10/31]] Lecture 3.2: Quantum noise models
\item[\bf [11/2]] Lecture 3.3: Nine-qubit quantum code
\item[\bf [11/7]] Lecture 3.4: Criterion for quantum codes
\item[\bf [11/9]] Lecture 3.5: Quantum CSS codes I
\item[\bf [11/14]] Lecture 3.6: Quantum CSS codes II
\item[\bf [11/16]] Lecture 3.7: Quantum key distribution
\end{itemize}

\noindent
{\bf Unit 4: Models of quantum computation}

\begin{itemize}
  \setlength\itemsep{0.1em}
\item[\bf [11/21]] Lecture 4.1: Measurement and quantum algorithms
\item[\bf [11/23]] Lecture 4.2: no class (Thanksgiving)
\item[\bf [11/28]] Lecture 4.3: Distributed quantum algorithms I
\item[\bf [11/28]] Lecture 4.4: Distributed quantum algorithms II
\item[\bf [11/30]] Lecture 4.5: Noise, computation, and fault-tolerance
\item[\bf [12/5]] Lecture 4.6: Quantum fault-tolerance
\item[\bf [12/7]] Lecture 4.7: Adiabatic quantum computing I
\item[\bf [12/12]] Lecture 4.8: Adiabatic quantum computing II
\end{itemize}

\begin{itemize}
\item[\bf [12/TBD]] Final exam
\end{itemize}

\end{document}
