
%%%%%%%%%%%%%%%%%%%%%%%%%%%%%%%%%%%%%%%%%%%%%%%%%%%%%%%%%%%%%%%%%%%%%%%%%%%%%

\begin{edXsection}{Introduction to 8.370r}

\begin{edXtext}{About this course}

{\LARGE About this course}

{\noindent\bf\Large Welcome to 8.370r Quantum Information Science I.}

This course rovides an introduction to the theory and practice of quantum
computation. We cover the physics of information processing;
quantum logic; quantum algorithms including Shor's factoring algorithm
and Grover's search algorithm; quantum error correction; quantum
communication and cryptography. Prior knowledge of quantum mechanics
helpful but not required. This is the first course in a sequence of two
quantum information science courses at MIT.

You may find helpful the
\href{https://www.edx.org/course/mastering-quantum-mechanics-part-1-wave-mitx-8-05-1x}{MIT
  8.05x courses on edX}, and Berkeley's
\href{https://www.edx.org/course/quantum-mechanics-quantum-computation-uc-berkeleyx-cs-191x}{CS191
  course}.  The
\href{https://www.edx.org/course/quantum-cryptography-caltechx-delftx-qucryptox}{Caltech-TU
  Delft course on quantum cryptography} may also be insightful.

This course comprises four units, each with lectures and
in-depth, automatically graded problems:
\begin{itemize}
\item{\href{/course/courseware/unit1}{Unit 1: Quantum and classical
    computing fundamentals}} -- an introduction to quantum computation;
  classical Boolean logic; introduction to quantum mechanics; and
  quantum wierdness.

\item{\href{/course/courseware/unit2}{Unit 2: Simple quantum protocols
    and algorithms}} -- covering quantum teleportation, superdense
  coding, the quantum circuit model, the Deutsch-Jozsa quantum
  algorithm, Simon's algorithm, the quantum Fourier transform, phase
  estimation, and Shor's quantum factoring algorith.

\item{\href{/course/courseware/unit3}{Unit 3: Quantum noise, codes,
    and communication}} -- illustrating classical error correction
  codes, quantum noise models, the nine-qubit quantum code, criteria
  for quantum codes, Calderbank-Shor-Steane codes, and quantum key
  distribution.

\item{\href{/course/courseware/unit4}{Unit 4: Models of quantum
    computation}} -- describing the role of measurement in quantum
  algorithms, distributed quantum algorithms, noisy computation and
  fault-tolerance, quantum fault-tolerance, and adiabatic quantum
  computation.

\end{itemize}

{\noindent\bf\Large Assessments and Deadlines}

There are eight problem sets - two for each unit.  

{\noindent\bf\Large Grading and Certificates}

The course grade is 40\% homework, 20\% midterm exam, 30\% final exam, and 10\% participation.

{\noindent\bf\Large Honor Code}

As described in the \href{https://www.edx.org/edx-terms-service}{edX Honor code}, you are expected to:

\begin{itemize}
  \item
    Complete all tests and assignments on my own, unless collaboration on an assignment is explicitly permitted.
  \item
    Maintain only one user account and not let anyone else use my username and/or password.
  \item
    Not engage in any activity that would dishonestly improve my results, or improve or hurt the results of others.
  \item
    Not post answers to problems that are being used to assess student performance.
\end{itemize}

{\noindent\bf\Large Textbook and References}

You may find it helpful to refer to:
\href{https://www.amazon.com/Quantum-Computation-Information-10th-Anniversary/dp/1107002176}{Quantum
  Computation and Quantum Information}, by Nielsen and Chuang.  There are also excellent, freely available 
\href{http://www.theory.caltech.edu/~preskill/ph219/index.html#lecture}{lecture notes by John Preskill}, and 
superb \href{http://pirsa.org/C15009}{video lectures by Daniel Gottesman}.

% \edXaskta{settings=1 to=csamolis@mit.edu cc=ichuang@mit.edu}

\end{edXtext}

\end{edXsection}

%%%%%%%%%%%%%%%%%%%%%%%%%%%%%%%%%%%%%%%%%%%%%%%%%%%%%%%%%%%%%%%%%%%%%%%%%%%%%
